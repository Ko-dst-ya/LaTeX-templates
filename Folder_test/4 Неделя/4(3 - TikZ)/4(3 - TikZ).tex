\documentclass[a4paper,12pt,leqno]{article}

%%% Работа с русским языком
\usepackage{cmap}					% поиск в PDF
\usepackage{mathtext} 				% русские буквы в фомулах
\usepackage[T2A]{fontenc}			% кодировка
\usepackage[utf8]{inputenc}			% кодировка исходного текста
\usepackage[english,russian]{babel}	% локализация и переносы

%%% Дополнительная работа с математикой
\usepackage{amsmath,amsfonts,amssymb,amsthm,mathtools} % AMS
\usepackage{icomma} % "Умная" запятая: $0,2$ --- число, $0, 2$ --- перечисление

%% Номера формул
%\mathtoolsset{showonlyrefs=true} % Показывать номера только у тех формул, на которые есть \eqref{} в тексте.
%\usepackage{leqno} % Немуреация формул слева

%% Свои команды
\DeclareMathOperator{\sgn}{\mathop{sgn}}

%% Перенос знаков в формулах (по Львовскому)
\newcommand*{\hm}[1]{#1\nobreak\discretionary{}
{\hbox{$\mathsurround=0pt #1$}}{}}

%%% Работа с картинками
\usepackage{graphicx}  % Для вставки рисунков
\graphicspath{{images/}{images2/}}  % папки с картинками
\setlength\fboxsep{3pt} % Отступ рамки \fbox{} от рисунка
\setlength\fboxrule{1pt} % Толщина линий рамки \fbox{}
\usepackage{wrapfig} % Обтекание рисунков текстом

%%% Работа с таблицами
\usepackage{array,tabularx,tabulary,booktabs} % Дополнительная работа с таблицами
\usepackage{longtable}  % Длинные таблицы
\usepackage{multirow} % Слияние строк в таблице

%%% Теоремы
\theoremstyle{plain} % Это стиль по умолчанию, его можно не переопределять.
\newtheorem{theorem}{Теорема}[section]
\newtheorem{proposition}[theorem]{Утверждение}
 
\theoremstyle{definition} % "Определение"
\newtheorem{corollary}{Следствие}[theorem]
\newtheorem{problem}{Задача}[section]
 
\theoremstyle{remark} % "Примечание"
\newtheorem*{nonum}{Решение}

%%% Программирование
%\usepackage{etoolbox} % логические операторы

%%% Страница
\usepackage{extsizes} % Возможность сделать 14-й шрифт
\usepackage{geometry} % Простой способ задавать поля
	\geometry{top=25mm}
	\geometry{bottom=35mm}
	\geometry{left=35mm}
	\geometry{right=20mm}
 %
%\usepackage{fancyhdr} % Колонтитулы
% 	\pagestyle{fancy}
 	%\renewcommand{\headrulewidth}{0pt}  % Толщина линейки, отчеркивающей верхний колонтитул
% 	\lfoot{Нижний левый}
% 	\rfoot{Нижний правый}
% 	\rhead{Верхний правый}
% 	\chead{Верхний в центре}
% 	\lhead{Верхний левый}
%	\cfoot{Нижний в центре} % По умолчанию здесь номер страницы

\usepackage{setspace} % Интерлиньяж
%\onehalfspacing % Интерлиньяж 1.5
%\doublespacing % Интерлиньяж 2
%\singlespacing % Интерлиньяж 1

\usepackage{lastpage} % Узнать, сколько всего страниц в документе.

\usepackage{soul} % Модификаторы начертания

\usepackage{hyperref}
\usepackage[usenames,dvipsnames,svgnames,table,rgb]{xcolor}
\hypersetup{				% Гиперссылки
    unicode=true,           % русские буквы в раздела PDF
    pdftitle={Заголовок},   % Заголовок
    pdfauthor={Автор},      % Автор
    pdfsubject={Тема},      % Тема
    pdfcreator={Создатель}, % Создатель
    pdfproducer={Производитель}, % Производитель
    pdfkeywords={keyword1} {key2} {key3}, % Ключевые слова
    colorlinks=true,       	% false: ссылки в рамках; true: цветные ссылки
    linkcolor=red,          % внутренние ссылки
    citecolor=black,        % на библиографию
    filecolor=magenta,      % на файлы
    urlcolor=cyan           % на URL
}

\usepackage{csquotes} % Еще инструменты для ссылок

%\usepackage[style=authoryear,maxcitenames=2,backend=biber,sorting=nty]{biblatex}  % Библиография

\usepackage{multicol} % Несколько колонок

%\usepackage{tikz} % Работа с графикой
\usepackage{pgfplots}
\usepackage{pgfplotstable}


\author{Библиография и Ti\emph{k}Z в \LaTeX{}\\
Первый документ}
\title{4 Неделя в \LaTeX{}}
\date{2 июля 2020}

%\documentclass[10pt]{article}
\usepackage{pgf,tikz,pgfplots}
\pgfplotsset{compat=1.15}
\usepackage{mathrsfs}
\usetikzlibrary{arrows}
\pagestyle{empty}

\begin{document}

\section{Игрульки с Ti\emph{k}Z}

\definecolor{qqwuqq}{rgb}{0,0.39215686274509803,0}
\begin{tikzpicture}[line cap=round,line join=round,>=triangle 45,x=1cm,y=1cm,scale=1.5]
\begin{axis}[
x=1cm,y=1cm,
axis lines=middle,
ymajorgrids=true,
xmajorgrids=true,
xmin=-3.321170133869765,
xmax=5.318819222246226,
ymin=-2.2555344419363337,
ymax=2.5050693064511367,
xtick={-3,-2,...,5},
ytick={-2,-1,...,2},]
\clip(-3.321170133869765,-2.2555344419363337) rectangle (5.318819222246226,2.5050693064511367);
\draw[line width=2pt,color=qqwuqq,smooth,samples=100,domain=0.1:5.318819222246226] plot(\x,{sin((1/(\x))*180/pi)});
\begin{scriptsize}
\draw[color=qqwuqq] (-3.230009636560218,-0.32597058221758246) node {$f$};
\end{scriptsize}
\end{axis}
\end{tikzpicture}

\vspace{10em}

\definecolor{myblue}{HTML}{92dcec}
\begin{center}
\begin{tikzpicture}[scale=2]
  
  \foreach \x/\y/\r/\color in {3.7 cm/ 1.3 cm/ 2 mm/ 80 ,
    3 cm/ 1.5 cm/ 1 mm/ 80 , 1.7 cm/ 0.4 cm/ 2 mm/ 30 ,
    0.3 cm/ 0.2 cm/ 2 mm/ 70 , 0.5 cm/ 0.4 cm/ 1 mm/ 80 ,
    2.4 cm/ 1.5 cm/ 1 mm/ 50 , 1.9 cm/ 1.4 cm/ 2 mm/ 10 ,
    4.7 cm/ 1.3 cm/ 2 mm/ 100 , 1.3 cm/ 0.8 cm/ 2 mm/ 100 ,
    2.8 cm/ 1.4 cm/ 1 mm/ 100 , 4.3 cm/ 0.2 cm/ 2 mm/ 100 ,
    0 cm/ 0.5 cm/ 2 mm/ 50 , 4.8 cm/ 0.7 cm/ 2 mm/ 60 ,
    2.7 cm/ 0 cm/ 1 mm/ 100 , 5 cm/ 0.9 cm/ 2 mm/ 50 ,
    1.3 cm/ 0.6 cm/ 3 mm/ 70 , 1 cm/ 1.2 cm/ 3 mm/ 70 ,
    0 cm/ 0.3 cm/ 2 mm/ 30 , 2.9 cm/ 0.2 cm/ 2 mm/ 100 ,
    5.5 cm/ 0.5 cm/ 2 mm/ 100 , 1.9 cm/ 0.2 cm/ 1 mm/ 100 ,
    2.3 cm/ 0.2 cm/ 1 mm/ 50 , 1.5 cm/ 1.3 cm/ 1 mm/ 50 ,
    3.5 cm/ 1.1 cm/ 2 mm/ 40 , 3.3 cm/ 0.1 cm/ 3 mm/ 100 ,
    4.7 cm/ 0.8 cm/ 2 mm/ 90 , 0.8 cm/ 0.3 cm/ 1 mm/ 100 , 
    0.7 cm/ 0.7 cm/ 2 mm/ 40 , 5.4 cm/ 0.8 cm/ 2 mm/ 70 ,
    0.8 cm/ 1.4 cm/ 3 mm/ 100 , 1.1 cm/ 1.1 cm/ 1 mm/ 40 ,
    2.5 cm/ 0.9 cm/ 2 mm/ 40 , 4.4 cm/ 1.4 cm/ 2 mm/ 60 , 
    1.3 cm/ 0.7 cm/ 1 mm/ 50 , 0.1 cm/ 0.2 cm/ 3 mm/ 90 ,
    4.1 cm/ 0.1 cm/ 2 mm/ 60 , 0.3 cm/ 1.4 cm/ 1 mm/ 60 ,
    1.5 cm/ 0.9 cm/ 2 mm/ 60 , 3.2 cm/ 1.4 cm/ 2 mm/ 80 ,
    4.8 cm/ 0.3 cm/ 2 mm/ 100 , 4.7 cm/ 0.9 cm/ 2 mm/ 50 ,
    4.7 cm/ 0.2 cm/ 2 mm/ 30 , 2.2 cm/ 0.3 cm/ 2 mm/ 70 ,
    3.6 cm/ 1.4 cm/ 1 mm/ 80 , 5.7 cm/ 0.5 cm/ 2 mm/ 50 ,
    4.3 cm/ 0.4 cm/ 3 mm/ 40 , 4 cm/ 0.6 cm/ 2 mm/ 10 ,
    5.5 cm/ 1 cm/ 1 mm/ 60 , 3.1 cm/ 0.3 cm/ 2 mm/ 10 ,
    4.6 cm/ 0.3 cm/ 2 mm/ 40
  }
    \shade[ball color = myblue!\color] (\x, \y) circle (\r);
\end{tikzpicture}
\end{center}

\begin{center}
\begin{tikzpicture}
	
	\draw[black,domain=0.01:3,smooth,thick] plot (\x, {sin (1/\x r)});
\end{tikzpicture}
\end{center}

\section{Тест по 4-ой неделе (Ti\emph{k}Z)}

\vspace{2em}

\begin{tikzpicture}
	\draw[<-] (3,2) -- (0,0) node {$Arrow$};
	\draw[domain=-5:5] plot (\x, \x/2 + 2);	
	\draw[domain=-5:5] plot (\x, \x*0.5 + 2);
	\draw[green] (0,0) -- (0,1) -- (1,0) -- (1,1) -- (0,0);
\end{tikzpicture}

\vspace{3em}
Для команды \verb"\draw" важен порядок необязательных аргументов:


\begin{figure}[h]
\begin{center}
	\begin{tikzpicture}[scale=2.5]
	\draw[fill=red,green] (1,1) circle [radius=1] node[black] {\small{\verb"\draw[fill=red,green]"}};
	\draw[green,fill=red] (4,1) circle [radius=1] node[yellow] {\small{\verb"\draw[green,fill=red]"}};
	\end{tikzpicture}
\end{center}
\end{figure}


\end{document}

