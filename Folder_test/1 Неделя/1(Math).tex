\documentclass[a4paper,12pt]{article}

\usepackage{cmap}					% поиск в PDF
\usepackage[T2A]{fontenc}			% кодировка
\usepackage[utf8]{inputenc}			% кодировка исходного текста
\usepackage[english,russian]{babel}	% локализация и переносы


%%% Дополнительная работа с математикой
\usepackage{amsmath,amsfonts,amssymb,amsthm,mathtools} % AMS
\usepackage{icomma} % "Умная" запятая: $0,2$ --- число, $0, 2$ --- перечисление

%% Номера формул
%\mathtoolsset{showonlyrefs=true} % Показывать номера только у тех формул, на которые есть \eqref{} в тексте.

%% Шрифты
\usepackage{euscript}	 % Шрифт Евклид
\usepackage{mathrsfs} % Красивый матшрифт

%% Свои команды
\DeclareMathOperator{\sgn}{\mathop{sgn}} % Функция знака sign (signum)
\DeclareMathOperator{\card}{\mathop{card}} % Мощность множества
\DeclareMathOperator{\im}{\mathop{Im}} % Образ отображения
\DeclareMathOperator{\divergence}{\mathop{div}} % Дивергенция
\DeclareMathOperator{\rot}{\mathop{rot}} % Ротор



%% Перенос знаков в формулах (по Львовскому)
\newcommand*{\hm}[1]{#1\nobreak\discretionary{}
{\hbox{$\mathsurround=0pt #1$}}{}}

%%%Заголовок
\author{Математика в \LaTeX{}}
\title{1 Неделя \LaTeX{}}
\date{26 июня 2020}

\begin{document} % Конец преамбулы, начало текста.

\maketitle



\section{Теоретико-множественные понятия}\label{sets}

$\aleph_{0}$ - обозначение мощности множества натуральных чисел $\mathbb{N}$:

\[ 
\card{\mathbb{N}} = \aleph_{0},
\]

где для произольного множества A $card{A}$ -- есть мощность A (от англ. cardinality -- мощность). Например, мощность пустого множества равна нулю:

\[ 
\card{\emptyset} = 0.
\]

Ещё примеры:

\[ 
\card{(\mathbb{N} \cup \mathbb{N})} = \aleph_{0},
\]

\[ 
\card{(\mathbb{N} \cap \mathbb{Q})} = \card{\mathbb{N}} = \aleph_{0},
\]

где $\mathbb{Q}$ - множество всех рациональных чисел.

А обозначение $2^{\aleph_{0}}$ используется для обозначения мощности множества всех подмножеств натуральных чисел и называется мощностью континуума. Часто вместо $2^{\aleph_{0}}$ пишут $\mathfrak{c}$. 
Что интересно, мощность множества действительных чисел в точности равна $\mathfrak{c}$:

\[
\card{\mathbb{R}} = 2^{\aleph_{0}} = \mathfrak{c}.
\]



\section{Математические формулы}\label{math_formulas}

Как думаете, чему равен предел такой суммы:

\begin{equation}\label{eq:lim_sum_pi_4}
	\lim_{n\to\infty} 
		n \left(\frac{1}{n^2 + 1} + \frac{1}{n^2 + 4} + \dots + \frac{1}{2n^2}
						\right),
\end{equation}

или, в эквивалентной форме:

\[ 
	\lim_{n\to\infty} n \sum_{i=1}^n \frac{1}{1+\left(\dfrac{i}{n}\right)^2}
\]

Он, на самом деле, как мы увилем в разделе \ref{integrals}, равен $\dfrac{\pi}{4}$.



\section{Математика в физике}\label{phys_and_math}

Известный закон Фика для диффузии лёгкой примеси в <<тяжёлом>> относительно примеси фоне записывается следующим образом (<<одномерный случай>>):

\begin{equation}%\label{eq:Fick's_Law}
j = -D\frac{\partial n}{\partial x},
\end{equation}

где j -- плотность потока примеси $\left[ \text{м}^{-2} \cdot \text{с}^{-1}\right]$.

Аналогичным образом выглядит закон Фурье для теплопроводности (<<одномерный случай>>):

\begin{equation}\label{eq:Fourier's_Law}
q = -\varkappa\frac{\partial T}{\partial x},
\end{equation}

где q -- плотность потока тепла $\left[ \text{Дж}\text{ м}^{-3}\right]$.

Из уравнения \eqref{eq:Fourier's_Law} легко получается уравнение теплопроводности:

\begin{equation}\label{eq:thermal_conductivity}
\rho C_{\textit{v}}^{(m)}\frac{\partial T}{\partial t} = \varkappa\frac{\partial T}{\partial x},
\end{equation}

где $C_{\textit{v}}^{(m)}$ -- удельная теплоёмкость вещества при постоянном объёме.

Это уравнение можно переписать следующим образом:

\[
\frac{\partial T}{\partial t} = \chi\frac{\partial T}{\partial x},
\]

$\chi$ называется температуропроводностью и имеет размерность коэффициента диффузии, что показывает родство процессов переноса (для вязкости можно ввести кинематическую вязкость с той же самой размерностью).

Рассмотрим систему уравнений Максвеллав в дифференциальной форме:

\[ \left\{
	\begin{aligned}
		\divergence \textbf{D} &= 4\pi\rho_{out} \\
		\rot \textbf{E} &= - \frac{1}{c} \frac{\partial \textbf{B}}{\partial t} \\
		\divergence \textbf{B} &= 0 \\
		\rot \textbf{H} &= \frac{4 \pi}{c} \textbf{j}_{out} + \frac{1}{c} \frac{\partial \textbf{D}}{\partial t}
	\end{aligned}
\right.
\]

\section{Интегралы}\label{integrals}

Рассмотрим функцию Римана:

\[
f_R(x) = \begin{cases}
	\cfrac{1}{q}, &\text{if } x = \cfrac{p}{q} \text{ where } p \in \mathbb{Z}, q \in \mathbb{N} \\
	0, &\text{if } x \in \mathbb{R} \backslash \mathbb{Q} 
\end{cases}
\]

Эта функция интересна тем, что $f_R \in R(\mathbb{R})$, то есть интегрируема по Риману на всей числовой прямой.
Этот факт можно доказать с помощью критерия Лебега интегрируемости по Риману (как упражнение покажите, что лебегова мера точек разрыва этой функции равна нулю).

Давайте объясним формулу \eqref{eq:lim_sum_pi_4}, которая нам встретилась на странице \pageref{eq:lim_sum_pi_4}.

\section{Матрицы}\label{Matrix} 

Смотри, матрица!!!!!
\[ 
\left|\left| 
\begin{array}{cc} 
1 & 2\\ 
3 & 4\\ 
5 & 6
\end{array} 
\right|\right| 
\]  

Представим, что у нас есть отображение $\varphi \in L(V, \tilde{V})$, которое в паре базисов e и f имеет матрицу A.
Рассмторим частный случай линейного отображения, а именно линейное преобразование ($\tilde{V} = V$). Пусть $\dim V = 3$ и $\varphi$ в паре базисов e и e имеет матрицу:
\[
	\begin{pmatrix}
		0 & 1 & 0 \\
		0 & 0 & 1 \\
		0 & 0 & 0 \\
	\end{pmatrix}
\]
Каков же геометрический смысл данного преобразования?
Вспомним смысл матрицы линейного преобразования: в столбцах стоят координаты базисных векторов новом (хотя в нашем случае он же и старый) базасе.
То есть одномерное подпространство $\langle\textbf{e}_1\rangle$ отобразилось в нуль, $\langle\textbf{e}_2\rangle$ перешло в $\langle\textbf{e}_1\rangle$, а $\langle\textbf{e}_3\rangle$ в $\langle\textbf{e}_2\rangle$.
Так мы можем с помощью геометрических представлений понять каково ядро и образ нашего преобразования $\varphi$:


\begin{align*}
	\ker \varphi &= \langle\textbf{e}_1\rangle,\\
	\im \varphi &= \langle\textbf{e}_1, \textbf{e}_2\rangle
.\end{align*}

Хороший способ вывода матрицы:

\[ 
\left( 
\begin{array}{cccc} 
a_{11} & a_{12} & \ldots & a_{1m}\\ 
a_{21} & a_{22} & \ldots & a_{2m}\\ 
\vdots & \vdots & \ddots & \vdots\\ 
a_{n1} & a_{n2} & \ldots & a_{nm} 
\end{array} 
\right) 
\] 

\[ 
\begin{pmatrix} 
a_{11} & a_{12} & \ldots & a_{1m} \\
a_{21} & a_{22} & \ldots & a_{2m} \\
\vdots & \vdots & \ddots & \vdots \\
a_{n1} & a_{n2} & \ldots & a_{nm} 
\end{pmatrix} 
\]

\end{document} % Конец текста.
