\documentclass[a4paper,12pt]{article}

%%% Работа с русским языком
\usepackage{cmap}					% поиск в PDF
%\usepackage{mathtext} 				% русские буквы в формулах
\usepackage[T2A]{fontenc}			% кодировка
\usepackage[utf8]{inputenc}			% кодировка исходного текста
\usepackage[english,russian]{babel}	% локализация и переносы
\usepackage{indentfirst}
\frenchspacing

\renewcommand{\epsilon}{\ensuremath{\varepsilon}}
\renewcommand{\phi}{\ensuremath{\varphi}}
\renewcommand{\kappa}{\ensuremath{\varkappa}}
\renewcommand{\le}{\ensuremath{\leqslant}}
\renewcommand{\leq}{\ensuremath{\leqslant}}
\renewcommand{\ge}{\ensuremath{\geqslant}}
\renewcommand{\geq}{\ensuremath{\geqslant}}
\renewcommand{\emptyset}{\varnothing}

%%% Дополнительная работа с математикой
\usepackage{amsmath,amsfonts,amssymb,amsthm,mathtools} % AMS
\usepackage{icomma} % "Умная" запятая: $0,2$ --- число, $0, 2$ --- перечисление
\usepackage{mathrsfs}

%% Номера формул
%\mathtoolsset{showonlyrefs=true} % Показывать номера только у тех формул, на которые есть \eqref{} в тексте.
%\usepackage{leqno} % Нумереация формул слева

%% Свои команды
\DeclareMathOperator{\sgn}{\mathop{sgn}}

%% Перенос знаков в формулах (по Львовскому)
\newcommand*{\hm}[1]{#1\nobreak\discretionary{}
{\hbox{$\mathsurround=0pt #1$}}{}}

%%% Работа с картинками
\usepackage{graphicx}  % Для вставки рисунков
\graphicspath{{images/}{images2/}}  % папки с картинками
\setlength\fboxsep{3pt} % Отступ рамки \fbox{} от рисунка
\setlength\fboxrule{1pt} % Толщина линий рамки \fbox{}
\usepackage{wrapfig} % Обтекание рисунков текстом

%%% Работа с таблицами
\usepackage{array,tabularx,tabulary,booktabs} % Дополнительная работа с таблицами
\usepackage{longtable}  % Длинные таблицы
\usepackage{multirow} % Слияние строк в таблице

%%% Теоремы
\theoremstyle{plain} % Это стиль по умолчанию, его можно не переопределять.
\newtheorem{theorem}{Теорема}[section]
\newtheorem{proposition}[theorem]{Утверждение}
 
\theoremstyle{definition} % "Определение"
\newtheorem{corollary}{Следствие}[theorem]
\newtheorem{problem}{Задача}[section]
 
\theoremstyle{remark} % "Примечание"
\newtheorem*{nonum}{Решение}

%%% Программирование
\usepackage{etoolbox} % логические операторы

%%% Страница
\usepackage{extsizes} % Возможность сделать 14-й шрифт
\usepackage{geometry} % Простой способ задавать поля
	\geometry{top=25mm}
	\geometry{bottom=35mm}
	\geometry{left=35mm}
	\geometry{right=20mm}
 %
%\usepackage{fancyhdr} % Колонтитулы
% 	\pagestyle{fancy}
 	%\renewcommand{\headrulewidth}{0pt}  % Толщина линейки, отчеркивающей верхний колонтитул
% 	\lfoot{Нижний левый}
% 	\rfoot{Нижний правый}
% 	\rhead{Верхний правый}
% 	\chead{Верхний в центре}
% 	\lhead{Верхний левый}
%	\cfoot{Нижний в центре} % По умолчанию здесь номер страницы

\usepackage{setspace} % Интерлиньяж
%\onehalfspacing % Интерлиньяж 1.5
%\doublespacing % Интерлиньяж 2
%\singlespacing % Интерлиньяж 1

\usepackage{lastpage} % Узнать, сколько всего страниц в документе.

\usepackage{soul} % Модификаторы начертания

\usepackage{hyperref}
\usepackage[usenames,dvipsnames,svgnames,table,rgb]{xcolor}
\hypersetup{				% Гиперссылки
    unicode=true,           % русские буквы в раздела PDF
    pdftitle={Заголовок},   % Заголовок
    pdfauthor={Автор},      % Автор
    pdfsubject={Тема},      % Тема
    pdfcreator={Создатель}, % Создатель
    pdfproducer={Производитель}, % Производитель
    pdfkeywords={keyword1} {key2} {key3}, % Ключевые слова
    colorlinks=true,       	% false: ссылки в рамках; true: цветные ссылки
    linkcolor=red,          % внутренние ссылки
    citecolor=black,        % на библиографию
    filecolor=magenta,      % на файлы
    urlcolor=cyan           % на URL
}

\usepackage{csquotes} % Еще инструменты для ссылок

%\usepackage[style=authoryear,maxcitenames=2,backend=biber,sorting=nty]{biblatex}

\usepackage{multicol} % Несколько колонок

\usepackage{tikz} % Работа с графикой
\usepackage{pgfplots}
\usepackage{pgfplotstable}

\author{}
\title{Латинские буквы в разных шрифтах}
\date{}

\begin{document}

\maketitle

\begin{tabular}{|c|c|c|c|c|}
\hline
Regular font & \verb"\mathfrak" & \verb"\mathscr" & \verb"\mathcal" &  \verb"\mathbb" \\
\hline 
A & $\mathfrak{A}$ & $\mathscr{A}$ & $\mathcal{A}$ & $\mathbb{A}$ \\ 
\hline 
B & $\mathfrak{B}$ & $\mathscr{B}$ & $\mathcal{B}$ & $\mathbb{B}$ \\ 
\hline 
C & $\mathfrak{C}$ & $\mathscr{C}$ & $\mathcal{C}$ & $\mathbb{C}$ \\ 
\hline 
D & $\mathfrak{D}$ & $\mathscr{D}$ & $\mathcal{D}$ & $\mathbb{D}$ \\ 
\hline 
E & $\mathfrak{E}$ & $\mathscr{E}$ & $\mathcal{E}$ & $\mathbb{E}$ \\ 
\hline 
F & $\mathfrak{F}$ & $\mathscr{F}$ & $\mathcal{F}$ & $\mathbb{F}$ \\ 
\hline 
G & $\mathfrak{G}$ & $\mathscr{G}$ & $\mathcal{G}$ &  $\mathbb{G}$ \\ 
\hline 
H & $\mathfrak{H}$ & $\mathscr{H}$ & $\mathcal{H}$ &  $\mathbb{H}$ \\ 
\hline 
I & $\mathfrak{I}$ & $\mathscr{I}$ & $\mathcal{I}$ &  $\mathbb{I}$ \\ 
\hline 
J & $\mathfrak{J}$ & $\mathscr{J}$ & $\mathcal{J}$ &  $\mathbb{G}$ \\ 
\hline 
K & $\mathfrak{K}$ & $\mathscr{K}$ & $\mathcal{K}$ &  $\mathbb{K}$ \\ 
\hline 
L & $\mathfrak{L}$ & $\mathscr{L}$ & $\mathcal{L}$ &  $\mathbb{L}$ \\ 
\hline 
M & $\mathfrak{M}$ & $\mathscr{M}$ & $\mathcal{M}$ &  $\mathbb{M}$ \\ 
\hline 
N & $\mathfrak{N}$ & $\mathscr{N}$ & $\mathcal{N}$ &  $\mathbb{N}$ \\ 
\hline 
O & $\mathfrak{O}$ & $\mathscr{O}$ & $\mathcal{O}$ &  $\mathbb{O}$ \\ 
\hline 
P & $\mathfrak{P}$ & $\mathscr{P}$ & $\mathcal{P}$ &  $\mathbb{P}$ \\ 
\hline 
Q & $\mathfrak{Q}$ & $\mathscr{Q}$ & $\mathcal{Q}$ &  $\mathbb{Q}$ \\ 
\hline 
R & $\mathfrak{R}$ & $\mathscr{R}$ & $\mathcal{R}$ &  $\mathbb{R}$ \\ 
\hline 
S & $\mathfrak{S}$ & $\mathscr{S}$ & $\mathcal{S}$ &  $\mathbb{S}$ \\ 
\hline 
T & $\mathfrak{T}$ & $\mathscr{T}$ & $\mathcal{T}$ &  $\mathbb{T}$ \\ 
\hline 
U & $\mathfrak{U}$ & $\mathscr{U}$ & $\mathcal{U}$ &  $\mathbb{U}$ \\ 
\hline 
V & $\mathfrak{V}$ & $\mathscr{V}$ & $\mathcal{V}$ &  $\mathbb{V}$ \\ 
\hline 
W & $\mathfrak{W}$ & $\mathscr{W}$ & $\mathcal{W}$ &  $\mathbb{W}$ \\ 
\hline 
X & $\mathfrak{X}$ & $\mathscr{X}$ & $\mathcal{X}$ &  $\mathbb{X}$ \\ 
\hline 
Y & $\mathfrak{Y}$ & $\mathscr{Y}$ & $\mathcal{Y}$ &  $\mathbb{Y}$ \\ 
\hline 
Z & $\mathfrak{Z}$ & $\mathscr{Z}$ & $\mathcal{Z}$ &  $\mathbb{Z}$ \\  
\hline 
\end{tabular} 
%\mathfrak{A B C D E F G H I J K L M N O P Q R S T U V W X Y Z}\\
%\mathscr{A B C D E F G H I J K L M N O P Q R S T U V W X Y Z}


\end{document}

