\documentclass[a4paper,12pt]{article}

%%% Работа с русским языком
\usepackage{cmap}					% поиск в PDF
%\usepackage{mathtext} 				% русские буквы в формулах
\usepackage[T2A]{fontenc}			% кодировка
\usepackage[utf8]{inputenc}			% кодировка исходного текста
\usepackage[english,russian]{babel}	% локализация и переносы

%%% Дополнительная работа с математикой
\usepackage{amsmath,amsfonts,amssymb,amsthm,mathtools} % AMS
\usepackage{icomma} % "Умная" запятая: $0,2$ --- число, $0, 2$ --- перечисление
\usepackage{mathrsfs} % Красивый шрифт

%% Номера формул
%\mathtoolsset{showonlyrefs=true} % Показывать номера только у тех формул, на которые есть \eqref{} в тексте.
%\usepackage{leqno} % Нумерация формул слева

%% Свои команды
\DeclareMathOperator{\sgn}{\mathop{sgn}} % Функция знака 
\DeclareMathOperator{\card}{\mathop{card}} % Мощность множества
\DeclareMathOperator{\im}{\mathop{Im}} % Образ отображения
\DeclareMathOperator{\divergence}{\mathop{div}} % Дивергенция
\DeclareMathOperator{\rot}{\mathop{rot}} % Ротор

%% Перенос знаков в формулах (по Львовскому)
\newcommand*{\hm}[1]{#1\nobreak\discretionary{}
{\hbox{$\mathsurround=0pt #1$}}{}}

%%% Работа с картинками
\usepackage{graphicx}  % Для вставки рисунков
\graphicspath{{images/}{images2/}}  % папки с картинками
\setlength\fboxsep{3pt} % Отступ рамки \fbox{} от рисунка
\setlength\fboxrule{1pt} % Толщина линий рамки \fbox{}
\usepackage{wrapfig} % Обтекание рисунков текстом

%%% Работа с таблицами
\usepackage{array,tabularx,tabulary,booktabs} % Дополнительная работа с таблицами
\usepackage{longtable}  % Длинные таблицы
\usepackage{multirow} % Слияние строк в таблице

%%% Теоремы
%\style{plain} % Это стиль по умолчанию, его можно не переопределять.
\newtheorem{theorem}{Теорема}[section]
\newtheorem{proposition}[theorem]{Утверждение}
 
\theoremstyle{definition} % "Определение"
\newtheorem{corollary}{Следствие}[theorem]
\newtheorem{problem}{Задача}[section]
 
\theoremstyle{remark} % "Примечание"
\newtheorem*{nonum}{Решение}

%%% Программирование
\usepackage{etoolbox} % логические операторы

%%%% Страница
%\usepackage{extsizes} % Возможность сделать 14-й шрифт
%\usepackage{geometry} % Простой способ задавать поля
%	\geometry{top=25mm}
%	\geometry{bottom=35mm}
%	\geometry{left=35mm}
%	\geometry{right=20mm}
% %
%\usepackage{fancyhdr} % Колонтитулы
% 	\pagestyle{fancy}
% 	\renewcommand{\headrulewidth}{0mm}  % Толщина линейки, отчеркивающей верхний колонтитул
% 	\lfoot{Нижний левый}
% 	\rfoot{Нижний правый}
% 	\rhead{Верхний правый}
% 	\chead{Верхний в центре}
% 	\lhead{Верхний левый}
% 	% \cfoot{Нижний в центре} % По умолчанию здесь номер страницы
%
%\usepackage{setspace} % Интерлиньяж
%%\onehalfspacing % Интерлиньяж 1.5
%%\doublespacing % Интерлиньяж 2
%%\singlespacing % Интерлиньяж 1
%
%\usepackage{lastpage} % Узнать, сколько всего страниц в документе.
%
%\usepackage{soul} % Модификаторы начертания
%
%\usepackage{hyperref}
%\usepackage[usenames,dvipsnames,svgnames,table,rgb]{xcolor}
%\hypersetup{				% Гиперссылки
%    unicode=true,           % русские буквы в раздела PDF
%    pdftitle={Заголовок},   % Заголовок
%    pdfauthor={Автор},      % Автор
%    pdfsubject={Тема},      % Тема
%    pdfcreator={Создатель}, % Создатель
%    pdfproducer={Производитель}, % Производитель
%    pdfkeywords={keyword1} {key2} {key3}, % Ключевые слова
%    colorlinks=true,       	% false: ссылки в рамках; true: цветные ссылки
%    linkcolor=red,          % внутренние ссылки
%    citecolor=green,        % на библиографию
%    filecolor=magenta,      % на файлы
%    urlcolor=cyan           % на URL
%}

%\renewcommand{\familydefault}{\sfdefault} % Начертание шрифта

\usepackage{fancyhdr}
\pagestyle{fancy}
\rfoot{\LaTeX{}} 

\usepackage{multicol} % Несколько колонок

\author{Счётчики и макрокоманды \LaTeX{}}
\title{3 Неделя \LaTeX{}}
\date{30 июня 2020}

\begin{document} % конец преамбулы, начало документа

\maketitle


% Счётчик
\newcounter{nc}[section]


% Шаблон для задач
\newcommand{\z}[1]{%

\addtocounter{nc}{1} % 1 арг. - счётчик; 2 - на сколько его надо увеличить
\textit{\textbf{Задача \thesection.\arabic{nc}.}}\\ #1 \\%
}



\renewcommand{\thesection}{\Roman{section}}


\section{Некоторые теоремы из математики}\label{sec:math_theorems}

% Шаблон: подмножество R
\newcommand{\subsetR}[1]{\ensuremath{%
#1 \subset \mathbb{R}}%
} % 1 арг. - название для LaTeX; 2 (необ.) - кол-во аргументов в новой команде (н.к); 3 - описание действия н.к., которая пользуется аргументами, \ensuremath позволяет не ставить значок $ для перехода в мат. реж., #N - N-ый арг. н.к. 


% Шаблон: подмножество R^n
\newcommand{\subsetRn}[1]{\ensuremath{%
#1 \subset \mathbb{R}^n}%
}


% Шаблон: Лебегова мера 0
\newcommand{\Lmz}[1]{\ensuremath{%
\mathscr{L}_1 (#1) = 0}%
}


% Шаблон для ссылки на номера страниц с формулами
\newcommand{\str}[1]{%
на стр. #1%
}

% Переобозначение 
\renewcommand{\le}{\leqslant}


\begin{theorem}[Критерий Лебега интегрируемости по Риману]\label{Lebesgue_and_Riemann}
	Функция f инегрируема по Риману на отрезке \subsetR{[a, b]} тогда и только тогда, когда:\\
	1) f ограничена; \\
	2) \Lmz{E}, где E - множество всех точек разрыва функции f на отрезке $[a, b]$.
\end{theorem}

Из теоремы \ref{Lebesgue_and_Riemann} вытекает следующее

\begin{corollary}
Функция
\begin{equation}
f(x) =
\left\{
	\begin{aligned}
		\sin \frac{1}{x}&, \  x \neq 0; \\
		0&, \  otherwise
	\end{aligned}
\right.
\end{equation}
интегрируема на любом отрезке $[a, b] \subset \mathbb{R}$.
\end{corollary}

\begin{nonum}
Очевидно, что $f(x)$ ограничена на $\forall$ \subsetR{[a, b]} единицей, т.е.: \\
\[
	|f(x)| \le 1, \text{ } \forall \ a \le b \in \mathbb{R}
\]
Если $ 0 \notin [a, b]$, то, очевидно, \Lmz{E} (т.к. $E = \varnothing$).\\
Иначе ($ 0 \in [a, b]$) $E = \{0\}$, но всё равно \Lmz{E}, т.к. лебегова мера не более чем счётного множества равна нулю.
\end{nonum}

\begin{corollary}
Функция
\[
f_R(x) = \begin{cases}
	\cfrac{1}{q}, &\text{if } x = \cfrac{p}{q} \text{ where } p \in \mathbb{Z}, q \in \mathbb{N} \\
	0, &\text{if } x \in \mathbb{R} \backslash \mathbb{Q} 
\end{cases}
\]

интегрируема на любом \subsetR{[a, b]}.
\end{corollary}

\z{Докажите это самостоятельно.}

Теорема \ref{Lebesgue_and_Riemann} (\str{\pageref{Lebesgue_and_Riemann}}) -- мощный аппарат для исследования интегрируемости по Риману в собственном смысле, ведь она отвечает на вопрос: <<Какие функции интегрируемы на отрезке по Риману, а какие нет?>>

\section{Разнообразные задачки}\label{sec:probs}

\newcommand{\znum}[2]{#1.#2}

Предложим читателю для размышления пару известных задачек.

\z{Предположим, что \subsetRn{A}. Известно, что $A'$ -- конечное множество.\\
Доказать, что $\card A \le \aleph_0$.\\
\textit{\underline{Напоминание:}} Для произвольного м-ва $A$ из метрического простванства $A'$ -- <<производное множество>>, т.е. мн-во всех предельных точек $A$.}\label{prob:sets}

\z{Смешали две кислоты: $Ac_1$ и $Ac_2$ в соотношении 1:1. В результате превращения образовались в соотношении 2:1 вещества:
серная кислота и газ $G$ с $\mu (G) \hm{=} 44 \text{ г}/ \text{моль}$.\\
Определить формулы кислот $Ac_1$ и $Ac_2$ и газа $G$.}\label{prob:chem_2_acids}



\section{Задачи с ответами}

\newbool{answers}

%\booltrue{answers} % достаточно закомментировать это, чтобы доступные ответы исчезли

% Шаблон для задач с ответами
\renewcommand{\z}[2][None]{%

\addtocounter{nc}{1}
\textit{\textbf{Задача \thesection.\arabic{nc}.}}\\ #2 \\%
\ifbool{answers}{\textit{Ответ:} #1.}{}\\
}


\z[$H_2N_2O_2, H_2S_2O_7, N_2O$]{См. химическую задачу из раздела \ref{sec:probs}.}

\z[Например подойдёт оператор $\varphi: \mathbb{R}^3 \rightarrow \mathbb{R}^3$ с матрицей $A = 
\begin{pmatrix}
0 & 1 & 0 \\
0 & 0 & 1 \\
0 & 0 & 0 \\
\end{pmatrix}
$]{Приведите пример линейного преобразвания $\varphi: \varphi^3 = 0$.}

\begin{problem}[HEDM]\label{HEDM}
	Текст задачи.
\end{problem}

Посмотрим внимательно на условие задачи \ref{HEDM}.

\end{document} % конец документа


