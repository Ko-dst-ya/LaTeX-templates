\documentclass[a4paper,12pt]{article}

%%% Работа с русским языком
\usepackage{cmap}					% поиск в PDF
\usepackage{mathtext} 				% русские буквы в формулах
\usepackage[T2A]{fontenc}			% кодировка
\usepackage[utf8]{inputenc}			% кодировка исходного текста
\usepackage[english,russian]{babel}	% локализация и переносы

%%% Дополнительная работа с математикой
\usepackage{amsmath,amsfonts,amssymb,amsthm,mathtools} % AMS
\usepackage{icomma} % "Умная" запятая: $0,2$ --- число, $0, 2$ --- перечисление

%% Номера формул
%\mathtoolsset{showonlyrefs=true} % Показывать номера только у тех формул, на которые есть \eqref{} в тексте.
%\usepackage{leqno} % Нумерация формул слева

%% Свои команды
\DeclareMathOperator{\sgn}{\mathop{sgn}}

%% Перенос знаков в формулах (по Львовскому)
\newcommand*{\hm}[1]{#1\nobreak\discretionary{}
{\hbox{$\mathsurround=0pt #1$}}{}}

%%% Работа с картинками
\usepackage{graphicx}  % Для вставки рисунков
\graphicspath{{images/}{images2/}}  % папки с картинками
\setlength\fboxsep{3pt} % Отступ рамки \fbox{} от рисунка
\setlength\fboxrule{1pt} % Толщина линий рамки \fbox{}
\usepackage{wrapfig} % Обтекание рисунков текстом

%%% Работа с таблицами
\usepackage{array,tabularx,tabulary,booktabs} % Дополнительная работа с таблицами
\usepackage{longtable}  % Длинные таблицы
\usepackage{multirow} % Слияние строк в таблице

%%% Теоремы
\theoremstyle{plain} % Это стиль по умолчанию, его можно не переопределять.
\newtheorem{theorem}{Теорема}[section]
\newtheorem{proposition}[theorem]{Утверждение}
 
\theoremstyle{definition} % "Определение"
\newtheorem{corollary}{Следствие}[theorem]
\newtheorem{problem}{Задача}[section]
 
\theoremstyle{remark} % "Примечание"
\newtheorem*{nonum}{Решение}

%%% Программирование
\usepackage{etoolbox} % логические операторы

%%% Страница
%\usepackage{extsizes} % Возможность сделать 14-й шрифт
\usepackage{geometry} % Простой способ задавать поля
	\geometry{top=25mm}
	\geometry{bottom=35mm}
	\geometry{left=35mm}
	\geometry{right=20mm}
 %
\usepackage{fancyhdr} % Колонтитулы
 	\pagestyle{fancy}
 	\renewcommand{\headrulewidth}{0mm}  % Толщина линейки, отчеркивающей верхний колонтитул
 	\lfoot{Нижний левый}
 	\rfoot{Нижний правый}
 	\rhead{Верхний правый}
 	\chead{Верхний в центре}
 	\lhead{Верхний левый}
 	% \cfoot{Нижний в центре} % По умолчанию здесь номер страницы

\usepackage{setspace} % Интерлиньяж
%\onehalfspacing % Интерлиньяж 1.5
%\doublespacing % Интерлиньяж 2
%\singlespacing % Интерлиньяж 1

\usepackage{lastpage} % Узнать, сколько всего страниц в документе.

\usepackage{soulutf8} % Модификаторы начертания

%%% Гиперссылки
\usepackage{hyperref}
\usepackage[usenames,dvipsnames,svgnames,table,rgb]{xcolor}
\hypersetup{				% Гиперссылки
    unicode=true,           % русские буквы в раздела PDF
    pdftitle={Заголовок},   % Заголовок
    pdfauthor={Автор},      % Автор
    pdfsubject={Тема},      % Тема
    pdfcreator={Создатель}, % Создатель
    pdfproducer={Производитель}, % Производитель
    pdfkeywords={keyword1} {key2} {key3}, % Ключевые слова
    colorlinks=true,       	% false: ссылки в рамках; true: цветные ссылки
    linkcolor=red,          % внутренние ссылки
    citecolor=green,        % на библиографию
    filecolor=magenta,      % на файлы
    urlcolor=Blue           % на URL
}

%\renewcommand{\familydefault}{\sfdefault} % Начертание шрифта (без засечек)

\usepackage{multicol} % Несколько колонок

\usepackage{pgfplots} %Графики

\usepackage{lscape} % Альбомная ориентация

\author{Оформление документа в целом в \LaTeX{}}
\title{3 Неделя в \LaTeX}
\date{1 июля 2020}

\begin{document}

\maketitle

\newpage

\section{Пару слов о МОШ и не только}

Тестируем команду \text{\emph{$\backslash emph$}}:

Я пишу текст с \emph{важным фрагментом, на который я хочу выделить особо, но в нём есть \emph{ещё более важный фрагмент} и я выделяю его по правилам типографии}. Поэтому-то \emph{более важный фрагмент} становится обычным в написании.

Дважды я участвовал в МОШ (московская олимпиада школьников) по химии. В 10-м классе я стал победителем, а в 11-м призёром третьей степени \dots \,да, стыдновато \dots \,На практический тур необходимо было сделать реферат и к реферату предлагался титульный лист со следующей страницы.

Кстати, для написания солова <<\so{РЕФЕРАТ}>> использовалась команда \text{\emph{$\backslash so$}}, однако это не единственный вариант. Можно исользовать команду \text{\emph{$\backslash ,$}}. Она создаёт маленький пробел между словами (буквами): <<Р\,Е\,Ф\,Е\,Р\,А\,Т>>.

Ссылка на этот \href{http://moschem.olimpiada.ru/news/244}{титульник} лежит здесь: \url{http://moschem.olimpiada.ru/news/244}. %\fref{C:\Users\user\Downloads\tit_list.doc}

На экспериментальный тур организаторы призывали принести:

\begin{itemize}
\item[$\star$] распечатанные из личного кабинета бланки для выполнения работы :
\begin{enumerate}
\item титульный лист - 1 шт.,
\item лист проверки - 1 шт.,
\item бланки для выполнения работы - 5 шт.; 
\end{enumerate}

\item документ, удостоверяющий личность; 

\item[$\bigtriangleup$] распечатанный реферат (принимаются и рукописные варианты);

\item[$\diamond$] лабораторный халат и лабораторные перчатки;

\item[$\circlearrowleft$] 2 черные гелевые ручки;

\item[$\clubsuit$] непрограммируемый калькулятор.\\
\end{itemize}

Некоторые классы органических соединений:
\begin{enumerate}
	\begin{multicols}{2}
		\item Углеводороды
		\item Галогенпроизводные \\углеводородов
		\item Кислоты
		\item Спирты
		\item Производные кислот
		\item Карбонильные соединения
		\item Амины
	\end{multicols}
\end{enumerate}

\newpage

\thispagestyle{empty}
\large{
\begin{center}
	\textbf{LXXVI Московская олимпиада школьников по химии \\ \vspace{1em}
	\Large{Заключительный этап}\\}
	25 марта 2018 года \\
		\vspace{2em}
		\vspace{1ex}
	Московский технологический университет (МИТХТ)\\
		\vspace{7em}
	\so{РЕФЕРАТ}\\
		\vspace{1em}
		\vspace{1ex}	
	\textrm{Тема «Химия меди и её соединений (получение, свойства, качественные реакции соединений меди в степени окисления +I и +II)»}
\end{center}
	\vspace{4.5em}
\begin{flushright}
%\noindent % без создания нового абзаца
Выполнил ученик  10  класса \\
МБОУ <<Лицей <<Физико-техническая школа>>\\
Кравцов Константин Вадимович
\end{flushright}

\begin{center}
\vfill
Город Обнинск
\end{center}
}

\newpage

\section{Парабола}

\begin{tikzpicture}
    \begin{axis}[
        xtick = {0,1,5},
        xticklabels = {{zero},$\alpha$,$\varphi$},
        extra x ticks = {-4,-2},
        extra x tick style = {
                            red,
                            font=\bfseries
                            },
    ]
    \addplot {x^2 - x +4};
    \end{axis}
\end{tikzpicture}

\newpage

\begin{landscape}
Lorem ipsum dolor sit amet, consectetur adipisicing elit, sed do eiusmod tempor incididunt ut labore et dolore magna aliqua.
\end{landscape} 

\end{document}
