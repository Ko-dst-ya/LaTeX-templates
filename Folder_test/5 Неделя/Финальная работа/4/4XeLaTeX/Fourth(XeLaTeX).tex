%!TEX TS-program = xelatex

\documentclass[a4paper,14pt]{article}

%%% Работа с русским языком
\usepackage[english,russian]{babel}   %% загружает пакет многоязыковой вёрстки
\usepackage{fontspec}      %% подготавливает загрузку шрифтов Open Type, True Type и др.
\defaultfontfeatures{Ligatures={TeX},Renderer=Basic}  %% свойства шрифтов по умолчанию
\setmainfont[Ligatures={TeX,Historic}]{Times New Roman} %% задаёт основной шрифт документа
\setsansfont{Comic Sans MS}                    %% задаёт шрифт без засечек
\setmonofont{Courier New}
\usepackage{indentfirst}
\frenchspacing


\renewcommand{\epsilon}{\ensuremath{\varepsilon}}
\renewcommand{\phi}{\ensuremath{\varphi}}
\renewcommand{\kappa}{\ensuremath{\varkappa}}
\renewcommand{\le}{\ensuremath{\leqslant}}
\renewcommand{\leq}{\ensuremath{\leqslant}}
\renewcommand{\ge}{\ensuremath{\geqslant}}
\renewcommand{\geq}{\ensuremath{\geqslant}}
\renewcommand{\emptyset}{\varnothing}

%%% Дополнительная работа с математикой
\usepackage{amsmath,amsfonts,amssymb,amsthm,mathtools} % AMS
\usepackage{icomma} % "Умная" запятая: $0,2$ --- число, $0, 2$ --- перечисление

%% Номера формул
%\mathtoolsset{showonlyrefs=true} % Показывать номера только у тех формул, на которые есть \eqref{} в тексте.
%\usepackage{leqno} % Нумерация формул слева

%% Свои команды
%\DeclareMathOperator{\sgn}{\mathop{sgn}}

%% Перенос знаков в формулах (по Львовскому)
\newcommand*{\hm}[1]{#1\nobreak\discretionary{}
{\hbox{$\mathsurround=0pt #1$}}{}}

%%% Работа с картинками
\usepackage{graphicx}  % Для вставки рисунков
\graphicspath{{images/}{images2/}}  % папки с картинками
\setlength\fboxsep{3pt} % Отступ рамки \fbox{} от рисунка
\setlength\fboxrule{1pt} % Толщина линий рамки \fbox{}
\usepackage{wrapfig} % Обтекание рисунков текстом

%%% Работа с таблицами
\usepackage{array,tabularx,tabulary,booktabs} % Дополнительная работа с таблицами
\usepackage{longtable}  % Длинные таблицы
\usepackage{multirow} % Слияние строк в таблице

%%% Теоремы
\theoremstyle{plain} % Это стиль по умолчанию, его можно не переопределять.
\newtheorem{theorem}{Теорема}[section]
\newtheorem{proposition}[theorem]{Утверждение}
 
\theoremstyle{definition} % "Определение"
\newtheorem{corollary}{Следствие}[theorem]
\newtheorem{problem}{Задача}[section]
 
\theoremstyle{remark} % "Примечание"
\newtheorem*{nonum}{Решение}

%%% Программирование
%\usepackage{etoolbox} % логические операторы


%%% Страница
\usepackage{extsizes} % Возможность сделать 14-й шрифт
\usepackage{geometry} % Простой способ задавать поля
	\geometry{top=30mm}
	\geometry{bottom=40mm}
	\geometry{left=30mm}
	\geometry{right=20mm}
 %
%\usepackage{fancyhdr} % Колонтитулы
% 	\pagestyle{fancy}
 	%\renewcommand{\headrulewidth}{0pt}  % Толщина линейки, отчеркивающей верхний колонтитул
% 	\lfoot{Нижний левый}
% 	\rfoot{Нижний правый}
% 	\rhead{Верхний правый}
% 	\chead{Верхний в центре}
% 	\lhead{Верхний левый}
%	\cfoot{Нижний в центре} % По умолчанию здесь номер страницы

\usepackage{setspace} % Интерлиньяж
\onehalfspacing % Интерлиньяж 1.5
%\doublespacing % Интерлиньяж 2
%\singlespacing % Интерлиньяж 1

\usepackage{lastpage} % Узнать, сколько всего страниц в документе.

\usepackage{soul} % Модификаторы начертания

\usepackage{hyperref}
\usepackage[usenames,dvipsnames,svgnames,table,rgb]{xcolor}
\hypersetup{				% Гиперссылки
    unicode=true,           % русские буквы в раздела PDF
    pdftitle={Заголовок},   % Заголовок
    pdfauthor={Автор},      % Автор
    pdfsubject={Тема},      % Тема
    pdfcreator={Создатель}, % Создатель
    pdfproducer={Производитель}, % Производитель
    pdfkeywords={keyword1} {key2} {key3}, % Ключевые слова
    colorlinks=true,       	% false: ссылки в рамках; true: цветные ссылки
    linkcolor=black,          % внутренние ссылки
    citecolor=black,        % на библиографию
    filecolor=magenta,      % на файлы
    urlcolor=cyan           % на URL
}

\usepackage{csquotes} % Еще инструменты для ссылок

%\usepackage[style=authoryear,maxcitenames=2,backend=biber,sorting=nty]{biblatex}

\usepackage{multicol} % Несколько колонок

\usepackage{tikz} % Работа с графикой
\usepackage{pgfplots}
\usepackage{pgfplotstable}

\renewcommand{\theenumi}{\bullet{enumi}}

\setcounter{secnumdepth}{0}

\begin{document}

\section*{Полярное сияние\footnote{Википедия. (2015). Полярное сияние — Википедия, свободная энциклопедия. [Online; accessed 10-января-2016]. Retrieved from \url{https://ru.wikipedia.org/?oldid=75221213}}}

\textbf{Полярное сияние (северное сияние)} "--- свечение (\href{https://ru.wikipedia.org/wiki/%D0%9B%D1%8E%D0%BC%D0%B8%D0%BD%D0%B5%D1%81%D1%86%D0%B5%D0%BD%D1%86%D0%B8%D1%8F}{люминесценция})
верхних слоёв \href{https://ru.wikipedia.org/wiki/%D0%9F%D0%BB%D0%B0%D0%BD%D0%B5%D1%82%D0%B0}{атмосфер планет}, обладающих \href{https://ru.wikipedia.org/wiki/%D0%9C%D0%B0%D0%B3%D0%BD%D0%B8%D1%82%D0%BE%D1%81%D1%84%D0%B5%D1%80%D0%B0}{магнитосферой}, вследствие
их взаимодействия с заряженными частицами \href{https://ru.wikipedia.org/wiki/%D0%A1%D0%BE%D0%BB%D0%BD%D0%B5%D1%87%D0%BD%D1%8B%D0%B9_%D0%B2%D0%B5%D1%82%D0%B5%D1%80}{солнечного ветра}.


\tableofcontents


\section{Природа полярных сияний}

В очень ограниченном участке верхней атмосферы сияния могут быть
вызваны низкоэнергичными заряженными частицами солнечного ветра,
попадающими в полярную ионосферу через северный и южный полярные \textbf{каспы}. В северном полушарии каспенные сияния можно наблюдать
над Шпицбергеном в околополуденные часы.

При столкновении энергичных частиц плазменного слоя с верхней
атмосферой происходит возбуждение атомов и молекул газов, входящих
в её состав. Излучение возбуждённых атомов в видимом диапазоне и
наблюдается как полярное сияние. Спектры полярных сияний зависят
от состава атмосфер планет: так, например, если для Земли наиболее
яркими являются линии излучения возбуждённых кислорода и азота в
видимом диапазоне, то для Юпитера — линии излучения водорода в
ультрафиолете.

Поскольку ионизация заряженными частицами происходит наиболее
эффективно в конце пути частицы и плотность атмосферы падает с увеличением высоты в соответствии с барометрической формулой, то высота
появлений полярных сияний достаточно сильно зависит от параметров
атмосферы планеты, так, для Земли с её достаточно сложным составом
атмосферы красное свечение кислорода наблюдается на высотах 200~"---
400 км, а совместное свечение азота и кислорода "--- на высоте $\sim$ 110 км.
Кроме того, эти факторы обусловливают и форму полярных сияний "---
размытая верхняя и достаточно резкая нижняя границы.


\end{document}

